%%% Макет страницы %%%
% Выставляем значения полей (ГОСТ 7.0.11-2011, 5.3.7)
\geometry{a4paper,top=2cm,bottom=2cm,left=2.5cm,right=1cm}

%%% Кодировки и шрифты %%%
\ifxetexorluatex
    \setmainlanguage[babelshorthands=true]{english}  % Язык по-умолчанию English с поддержкой приятных команд пакета babel                
    \ifXeTeX
        \defaultfontfeatures{Ligatures=TeX,Mapping=tex-text}
    \else
        \defaultfontfeatures{Ligatures=TeX}
    \fi
    \setmainfont{Times New Roman}
    \newfontfamily\cyrillicfont{Times New Roman}
    \setsansfont{Arial}
    \newfontfamily\cyrillicfontsf{Arial}
    \setmonofont{Courier New}
    \newfontfamily\cyrillicfonttt{Courier New}
\else
    \IfFileExists{pscyr.sty}{\renewcommand{\rmdefault}{ftm}}{}
\fi

%%% Интервалы %%%
%linespread-реализация ближе к реализации полуторного интервала в ворде.
%setspace реализация заточена под шрифты 10, 11, 12pt, под остальные кегли хуже, но всё же ближе к типографской классике. 
%\linespread{1.3}                    % Полуторный интервал (ГОСТ Р 7.0.11-2011, 5.3.6)

%%% Выравнивание и переносы %%%
\sloppy                             % Избавляемся от переполнений
\clubpenalty=10000                  % Запрещаем разрыв страницы после первой строки абзаца
\widowpenalty=10000                 % Запрещаем разрыв страницы после последней строки абзаца

%%% Изображения %%%
\graphicspath{{images/}}            % Пути к изображениям

%%% Подписи %%%
\captionsetup{%
singlelinecheck=off,                % Многострочные подписи, например у таблиц
skip=2pt,                           % Вертикальная отбивка между подписью и содержимым рисунка или таблицы определяется ключом
justification=centering,            % Центрирование подписей, заданных командой \caption
}

%\precaption{\tabcapalign} % всегда идет перед подписью или \legend
%\captionnamefont{\normalfont\normalsize} % Шрифт надписи «Таблица #»; также определяет шрифт у \legend
%\captiondelim{\tablabelsep} % разделитель идентификатора с номером от наименования
%\captionstyle[\tabtitalign]{\tabtitalign}
%\captiontitlefont{\normalfont\normalsize} % Шрифт с текстом подписи

%%% Рисунки %%%
\DeclareCaptionLabelSeparator*{emdash}{~--- }             % (ГОСТ 2.105, 4.3.1)
\captionsetup[figure]{labelsep=emdash,font=onehalfspacing,position=bottom}

%%% Таблицы %%%
\DeclareCaptionFormat{tablecaption}{\raggedleft #1#2%   % Идентификатор таблицы справа, на отдельной строке
    #3}                                       % Наименование таблицы строкой ниже и центрировано, без переносов
\DeclareCaptionFormat{tablenocaption}{\raggedleft #1#2%   % Идентификатор таблицы справа, на отдельной строке
}                                                         % Наименование таблицы отсутствует
\captionsetup[table]{format=tablecaption,labelsep=emdash,singlelinecheck=off,font=onehalfspacing,position=top}  % пробельный разделитьель идентификатора с номером от наименования, многострочные наименования и прочее
\DeclareCaptionLabelFormat{continued}{Сontinue of the table~#2}

%%% Подписи подрисунков %%%
\renewcommand{\thesubfigure}{\asbuk{subfigure}}           % Буквенные номера подрисунков
\captionsetup[subfigure]{font={normalsize},               % Шрифт подписи названий подрисунков (не отличается от основного)
    labelformat=brace,                                    % Формат обозначения подрисунка
    justification=centering,                              % Выключка подписей (форматирование), один из вариантов            
}
%\DeclareCaptionFont{font12pt}{\fontsize{12pt}{13pt}\selectfont} % объявляем шрифт 12pt для использования в подписях, тут же надо интерлиньяж объявлять, если не наследуется
%\captionsetup[subfigure]{font={font12pt}}                 % Шрифт подписи названий подрисунков (всегда 12pt)

%%% Цвета гиперссылок %%%
\definecolor{linkcolor}{rgb}{0.9,0,0}
\definecolor{citecolor}{rgb}{0,0.6,0}
\definecolor{urlcolor}{rgb}{0,0,1}

%%% Настройки гиперссылок %%%
\hypersetup{
    linktocpage=true,           % ссылки с номера страницы в оглавлении, списке таблиц и списке рисунков
%    pdfpagelabels=false,        % set PDF page labels (true|false)
    plainpages=false,           % Forces page anchors to be named by the Arabic form  of the page number, rather than the formatted form
    colorlinks,                 % ссылки отображаются раскрашенным текстом, а не раскрашенным прямоугольником, вокруг текста
    linkcolor={linkcolor},      % цвет ссылок типа ref, eqref и подобных
    citecolor={citecolor},      % цвет ссылок-цитат
    urlcolor={urlcolor},        % цвет гиперссылок
}

\ifLuaTeX
    \hypersetup{
        unicode,                % Unicode encoded PDF strings
    }
\fi

%%% Шаблон %%%
\DeclareRobustCommand{\todo}{\textcolor{red}}       % решаем проблему превращения названия цвета в результате \MakeUppercase, http://tex.stackexchange.com/a/187930/79756 , \DeclareRobustCommand protects \todo from expanding inside \MakeUppercase
\setlength{\parindent}{2.5em}                       % Абзацный отступ. Должен быть одинаковым по всему тексту и равен пяти знакам (ГОСТ Р 7.0.11-2011, 5.3.7).

%%% Списки %%%
% Используем дефис для ненумерованных списков (ГОСТ 2.105-95, 4.1.7)
\renewcommand{\labelitemi}{\normalfont\bfseries{--}} 
\setlist{nosep,%                                    % Единый стиль для всех списков (пакет enumitem), без дополнительных интервалов.
    labelindent=\parindent,leftmargin=*%            % Каждый пункт, подпункт и перечисление записывают с абзацного отступа (ГОСТ 2.105-95, 4.1.8)
}
