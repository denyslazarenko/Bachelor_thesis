\chapter*{Introduction}							% Заголовок
\addcontentsline{toc}{chapter}{Introduction}	% Добавляем его в оглавление

\newcommand{\actuality}{}
\newcommand{\aim}{\textbf{Целью}}
\newcommand{\tasks}{задачи}
\newcommand{\defpositions}{\textbf{Основные положения, выносимые на~защиту:}}
\newcommand{\novelty}{\textbf{Научная новизна:}}
\newcommand{\influence}{\textbf{Научная и практическая значимость}}
\newcommand{\reliability}{\textbf{Степень достоверности}}
\newcommand{\probation}{\textbf{Апробация работы.}}
\newcommand{\contribution}{\textbf{Личный вклад.}}
\newcommand{\publications}{\textbf{Публикации.}}

{\actuality}

Nowadays, retail e-commerce sales are quickly increasing. A large online e-commerce
websites serve millions of users’ requests per day. Therefore it is necessary to make the processes of registrations and purchases as much convenient and fast as possible. For many classified platforms such as Amazon or Avito users who would like to create a new advertisement must to fill in the required fields: title, description, price and category. Choosing a category can be a tricky moment because in most cases users have to make a choice from more than hundred categories. Therefore, the problem  of advertisement automatic category prediction is very important in terms of saving moderators' time and as a result, decreasing the number of necessary moderators to process them. The effective algorithms which would work with text data, have high accuracy and an appropriate speed are in high demand.

{\aim} of this thesis is building an effective model which have high accuracy and an appropriate speed for classification of advertisements at e-commerce platform Jiji.ng. 
In particular:
\begin{enumerate}
	\item consider different models that are used for texts classification 
	\item compare performance of Deep Learning models 
	\item prove the efficiency of Convoloutional neural netwoks for NLP related tasks 
\end{enumerate}

%Необхідно буд {\tasks}:
%\begin{enumerate}
%\item Исследовать, разработать, вычислить и~т.\:д. и~т.\:п.
%\item Исследовать, разработать, вычислить и~т.\:д. и~т.\:п.
%\item Исследовать, разработать, вычислить и~т.\:д. и~т.\:п.
%\item Исследовать, разработать, вычислить и~т.\:д. и~т.\:п.
%\end{enumerate}

{\novelty}
Consideration and implementation of Deep Neural Networks for solving text classification tasks, search of the most efficient architecture and parameters. Analysis of the best model and results.

%
%
%\influence\ \ldots
%
%\reliability\ полученных результатов обеспечивается \ldots \ Результаты находятся в соответствии с результатами, полученными другими авторами.
%
%\probation\
%Основные результаты работы докладывались~на:
%перечисление основных конференций, симпозиумов и~т.\:п.
%
%\contribution\ Автор принимал активное участие \ldots
%
%\publications\ Основные результаты по теме диссертации изложены в ХХ печатных изданиях~\cite{Sokolov,Gaidaenko,Lermontov,Management},
%Х из которых изданы в журналах, рекомендованных ВАК~\cite{Sokolov,Gaidaenko}, 
%ХХ --- в тезисах докладов~\cite{Lermontov,Management}.
 % Характеристика работы по структуре во введении и в автореферате не отличается (ГОСТ Р 7.0.11, пункты 5.3.1 и 9.2.1), потому её загружаем из одного и того же внешнего файла, предварительно задав форму выделения некоторым параметрам

%% регистрируем счётчики в системе totcounter
\regtotcounter{totalcount@figure}
\regtotcounter{totalcount@table}       % Если поставить в преамбуле то ошибка в числе таблиц
\regtotcounter{TotPages}               % Если поставить в преамбуле то ошибка в числе страниц

\textbf{Объем и структура работы.} Диссертация состоит из~введения, четырёх глав, заключения и~двух приложений.
%% на случай ошибок оставляю исходный кусок на месте, закомментированным
%Полный объём диссертации составляет  \ref*{TotPages}~страницу с~\totalfigures{}~рисунками и~\totaltables{}~таблицами. Список литературы содержит \total{citenum}~наименований.
%
Полный объём диссертации составляет \formbytotal{TotPages}{страниц}{у}{ы}{} 
с~\formbytotal{totalcount@figure}{рисунк}{ом}{ами}{ами}
и~\formbytotal{totalcount@table}{таблиц}{ей}{ами}{ами}. Список литературы содержит  
\formbytotal{citenum}{наименован}{ие}{ия}{ий}.
