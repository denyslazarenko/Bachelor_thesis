\chapter*{CONCLUSION}						% Заголовок
\addcontentsline{toc}{chapter}{CONCLUSION}	% Добавляем его в оглавление

%% Согласно ГОСТ Р 7.0.11-2011:
%% 5.3.3 В заключении диссертации излагают итоги выполненного исследования, рекомендации, перспективы дальнейшей разработки темы.
%% 9.2.3 В заключении автореферата диссертации излагают итоги данного исследования, рекомендации и перспективы дальнейшей разработки темы.
%% Поэтому имеет смысл сделать эту часть общей и загрузить из одного файла в автореферат и в диссертацию:

%Основные результаты работы заключаются в следующем.
%%% Согласно ГОСТ Р 7.0.11-2011:
%% 5.3.3 В заключении диссертации излагают итоги выполненного исследования, рекомендации, перспективы дальнейшей разработки темы.
%% 9.2.3 В заключении автореферата диссертации излагают итоги данного исследования, рекомендации и перспективы дальнейшей разработки темы.
\begin{enumerate}
  \item На основе анализа \ldots
  \item Численные исследования показали, что \ldots
  \item Математическое моделирование показало \ldots
  \item Для выполнения поставленных задач был создан \ldots
\end{enumerate}

%И какая-нибудь заключающая фраза.
Ієрархічна кластеризація допомогла отримати кластера, які відрізняються
більше поведінковими характеристиками (глибина / ширина перегляду,
інтервали між подіями, кількість сесій, тощо), в той час як класичний Kmeans
більше відзначав характеристики, які стосуються пошуку людини (конкретні
хвороби та країни), та мав більший набір характеристик, що його відрізняють.
Насправді обидва методи можуть дозволити отримати типаж людини, тому що
лікарі говорять про пряму залежність між хворобами та поведінкою людини, але
для цього потрібна додаткова експертиза.
Примітно те, що отримані кластери дуже схожі один на інший. Також при
обидва методи виділили в особливий кластер групу користувачів, яка цікавиться
штучним запліднення в бюджетних країнах, набори яких перетинаються.
Оскільки результати кожної моделі або перетинаються, або входять один в
одного, то можна зробити висновок про те, що дані хороші та залежності в них
є, а то яка модель більш зручна і таргетованою більше залежить від бізнесу. В
даному випадку критерієм було власне визначення корисності кластеризації.
За всіма метриками кращою моделлю виявилась кластеризація Kmeans з
дванадцятьма кластерами. Вона може допомогти з визначенням рівня
усвідомлення та інформованості користувача, що дозволить раціонально
використовувати час координаторів: більш усвідомлені та поінформовані будуть
рекомендуватися старшим координаторам, а ті клієнти, інформацію про яких ще
потрібно уточнити, - молодшим.
Надалі
можна
спробувати
виділити
більше
характеристик
та
відслідковувати більше подій для більше детального опису користувача. Ще
одним варіантом розвитку є автоматизація даного процесу і розробка
програмного забезпечення, яка дозволяла кінцевому користувачеві отримувати
кілька запропонованих варіантів сегментації та вибирати в залежності від своїх
потреб, як в даному прикладі.