%%% Переопределение именований %%%
\renewcommand{\abstractname}{Аннотация}
\renewcommand{\alsoname}{см. также}
\renewcommand{\bibname}{Список литературы} % (ГОСТ Р 7.0.11-2011, 4)
\renewcommand{\ccname}{исх.}
\renewcommand{\contentsname}{Оглавление} % (ГОСТ Р 7.0.11-2011, 4)
\renewcommand{\enclname}{вкл.}
\renewcommand{\figurename}{Figure} % (ГОСТ Р 7.0.11-2011, 5.3.9)
\renewcommand{\headtoname}{вх.}
\renewcommand{\indexname}{Предметный указатель}
\renewcommand{\listfigurename}{Список рисунков}
\renewcommand{\listtablename}{Список таблиц}
\renewcommand{\pagename}{p.}
\renewcommand{\partname}{Часть}
\renewcommand{\refname}{Список литературы} % (ГОСТ Р 7.0.11-2011, 4)
\renewcommand{\seename}{см.}
\renewcommand{\tablename}{Table} % (ГОСТ Р 7.0.11-2011, 5.3.10)